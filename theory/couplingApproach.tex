The classic low Mach implementation uses an incremental approximate 
pressure projection scheme in which nonlinear convergence is obtained 
using outer Picard loops. Recently a full study on coupling approaches 
has been conducted using ASC Algorithm funds. In this project, coupling 
methods ranging from fully implicit, fully coupled equal order 
pressure/velocity interpolation with pressure stabilization to 
explicit advection/diffusion pressure projection schemes. A brief summary of the 
results follows. 

Specifically, five algorithms
were considered and are as follows: 1) a monolithic scheme in which 
advection and diffusion are implicit using full analytical sensitivities, 
2) monolithic momentum solve with implicit advection/diffusion
in the context of a fourth order stabilized incremental pressure projection
scheme, 3) monolithic momentum solve with explicit advection; implicit diffusion
in the context of a fourth order stabilized incremental pressure projection
scheme, 4) segregated momentum solve with implicit advection/diffusion in the
context of a fourth order stabilized incremental pressure porjectin scheme, and 
5) explicit momementum advection/diffusion predictor/corrector scheme in
the context of a second order stabilized pressure-free approximate projection
scheme.

Each of the above algorithms has been run in the context of a transient 
uniform flow low Mach flow. The emphasis of this project is transient flows. 
As such, the numbers below are to be cast in this context. If steady flows
are desired, conclusions may be different. The slowdown of each implementation
is relative to the core low Mach algorith, i.e., algorithm (4) above. Numbers less
than unity represent a speed-up whereas numbers greater than unity represent
a slow down: 1) 3.4x, 2) 1.2x, 3) 0.6x, 4) 1.0x, 5) 0.7x.

The above runs were made using a time step that corresponded to a CFL of slightly
less than unity. In this particlar flow, a transitionally turbulent open jet,
the diffusion time scale stability limit was not a factor. In other words, there
existed no detailed boundary layer at the wall bounded flow at the ground 
plane. Results for a Reynolds number of ~45K back step also are similar to the above 
jet results. 

In general, although a mixture of implicit diffusion and explicit advection
seem to be the winning combination, this scheme is very sensitive to
time step and must be used by an educated user. In general, the conclusions
are, thus far, that the standard segregated pressure projection scheme
is preferred.

The algorithm implemented in Nalu is a fourth order approximate projection scheme
with monolithic momentum coupling. Evaluation of a predictor/corrector approach
for reating flow is anticipated in the late FY15 time frame.

\section{Errors due to Splitting and Stabilization}

As noted in many of our papers, the error in the above
method can be written in block form (let's relax the variable
density nuance - or simple fold these extra terms into 
our operators). Here we specifically partition error into
both splitting (the pressure projection aspect of the alg that
factorizes the fully coupled system) and pressure stabilization. 
Note that when we run fully coupled simulation with the same
pressure stabilization algorithm, the answers converge to the
same result.

Below, also forgive the specific definitions of $\tau$. In general,
they represent a choice of projection and stabilization
time scales. Finally, the Laplace operator, e.g., ${\bf L_2}$, have the
taus built into them.

\begin{equation}
  \left[
    \begin{array}{lr}
      {\bf A}  &  {\bf G}  \\
      {\bf D}  &  {\bf 0}
    \end{array}
  \right]
%
  \left[
    \begin{array}{l}
      {\bf u}^{n+1}  \\
      p^{n+1} 
    \end{array}
  \right] =
%
  \left[
    \begin{array}{l}
      {\bf f}  \\
      0
    \end{array}
  \right]    + 
   \left[
    \begin{array}{l}
      ({\bf I}- \tau {\bf A } ){\bf G}(p^{n+1}-p^n) \\ 
      \epsilon({\bf L_i},\tau_i, {\bf D}, {\bf G})
  \end{array}
  \right] 
     \label{smoothSplitError}
\end{equation}


\noindent
where the error term that appears for the discrete continuity solve is given
by,

\begin{equation}
\epsilon({\bf L_i},\tau_i,{\bf D},{\bf G}) = (({\bf L_1}-{\bf D}\tau_3{\bf G}) \\ -({\bf L_2}-{\bf D}\tau_2{\bf G}))(p^{n+1}-p^{n}) \\ + ({\bf L_2}-{\bf D}\tau_2{\bf G})p^{n+1}
\label{contErrorDefined}
\end{equation}
%
For the sake of this write-up, let ${\bf L_1} = {\bf L_2}$ and 
$\tau_2 = \tau_3$.
