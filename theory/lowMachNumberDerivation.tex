The low Mach number equations are a subset of the fully compressible
equations of motion (momentum, continuity and energy), admitting large
variations in gas density while remaining acoustically incompressible.
The low Mach number equations are preferred over the full compressible
equations for low speed flow problems as the accoustics are of 
little consequence to the overall simulation accuracy. The technique avoids 
the need to resolve fast-moving acoustic signals.  
Derivations of the low Mach number equations can be found in 
found in Rehm and Baum,~\cite{Rehm:1978}, or Paolucci,~\cite{Paolucci:1982}. 

The equations are derived from
the compressible equations using a perturbation expansion in terms
of the lower limit of the Mach number squared; hence the name.  The
asymptotic expansion leads to a splitting of pressure
into a spatially constant thermodynamic pressure and a locally
varying dynamic pressure.  The dynamic pressure is decoupled from
the thermodynamic state and cannot propagate acoustic waves.  The
thermodynamic pressure is used in the equation of state and
to determine thermophysical properties.  The thermodynamic pressure
can vary in time and can be calculated using a global energy balance.

\subsection{Asymptotic Expansion}
\label{asymExp}

The asymptotic expansion for the low Mach number equations
begins with
the full compressible equations in Cartesian coordinates.  
The equations are the minimum set required to propagate
acoustic waves.  The equations
are written in divergence form using Einstein notation
(summation over repeated indices):
%
\begin{eqnarray}
{{\partial \rho} \over {\partial t}} + {{\partial \rho u_j} 
                               \over {\partial x_j}} & =  & 0 , \\
{{\partial \rho u_i} \over {\partial t}} + {{\partial \rho u_j u_i}
    \over {\partial x_j}} + {{\partial P} \over {\partial x_i}} & = &
    {{\partial \tau_{ij}} \over {\partial x_j}} + \rho g_i , \\
{{\partial \rho E} \over {\partial t}} + {{\partial \rho u_j H}
    \over {\partial x_j}}  & = &
    - {{\partial q_j} \over {\partial x_j}}
    + {{\partial u_i \tau_{ij}} \over {\partial x_j}} + \rho u_i g_i .
\end{eqnarray}
%
The primitive variables are the velocity components, $u_i$, the pressure,
$P$, and the temperature $T$.  The viscous shear stress tensor is
$\tau_{ij}$, the heat conduction is $q_i$, the total enthalpy is
$H$, the total internal energy is $E$, the density is $\rho$, and
the gravity vector is $g_i$.  The total internal energy and total
enthalpy contain the kinetic energy contributions.
The equations are closed using the following models and definitions:
%
\begin{eqnarray}
P & = & \rho {R \over W} T , \\
E & = & H - P/\rho , \\
H & = & h + {1 \over 2} u_k u_k , \\
\tau_{ij} & = & \mu \left( {{\partial u_i} \over {\partial x_j}}
                  +        {{\partial u_j} \over {\partial x_i}} \right)
            - {2 \over 3} \mu {{\partial u_k} \over {\partial x_k}} 
               \delta_{ij} , \\
q_i & = & - k {{\partial T} \over {\partial x_i}} .
\end{eqnarray}
%
The mean molecular weight of the gas is $W$, the molecular
viscosity is $\mu$, and the thermal conductivity is $k$.
A Newtonian fluid is assumed along with the Stokes hypothesis
for the stress tensor.

The equations are scaled so that the variables are all of order one.  
The velocities, lengths, and times are nondimensionalized by 
a characteristic velocity, $U_\infty$, and a length scale, $L$.  
The pressure, density, and temperature are nondimensionalized by $P_\infty$,
$\rho_\infty$, and $T_\infty$.  The enthalpy and energy are nondimensionalized
by $C_{p,\infty} T_\infty$.  Dimensionless variables are noted
by overbars.  The dimensionless equations are:
%
\begin{eqnarray}
{{\partial \bar{\rho}} \over {\partial \bar{t}}} 
    + {{\partial \bar{\rho} \bar{u}_j} 
       \over {\partial \bar{x}_j}} & =  & 0 , \\
{{\partial \bar{\rho} \bar{u}_i} \over {\partial \bar{t}}} 
  + {{\partial \bar{\rho} \bar{u}_j \bar{u}_i}
    \over {\partial \bar{x}_j}} + {1 \over {\gamma {\rm Ma}^2}}
      {{\partial \bar{P}} \over {\partial \bar{x}_i}} & = &
    {1 \over {\rm Re}}{{\partial \bar{\tau}_{ij}} 
            \over {\partial \bar{x}_j}} 
     +  {1 \over {\rm Fr}_i} \bar{\rho} , \\
{{\partial \bar{\rho} \bar{h}} \over {\partial \bar{t}}} 
   + {{\partial \bar{\rho} \bar{u}_j \bar{h}}
    \over {\partial \bar{x}_j}}  
  & = &
    - {1 \over {\rm Pr}} {1 \over {\rm Re}} 
                    {{\partial \bar{q}_j} \over {\partial \bar{x}_j}}
 + {{\gamma - 1} \over \gamma} {{\partial \bar{P}} \over  {\partial \bar{t}}} \\
   & + & {{\gamma - 1} \over \gamma} {{{\rm Ma}^2} \over {\rm Re}}
  {{\partial \bar{u}_i \bar{\tau}_{ij}} \over {\partial \bar{x}_j}}  
 +  \bar{\rho} \bar{u}_i {{\gamma - 1} \over \gamma} 
            {{{\rm Ma}^2} \over {\rm Fr}_i} \nonumber \\
  & - & {{\gamma - 1} \over 2} {\rm Ma}^2
 \left( {{\partial \bar{\rho} \bar{u}_k \bar{u}_k} \over {\partial \bar{t}}} 
    +   {{\partial \bar{\rho} \bar{u}_j \bar{u}_k \bar{u}_k} 
        \over {\partial \bar{x}_j}} \right) . \nonumber
\end{eqnarray}
%
The groupings of characteristic scaling terms are:
%
\begin{eqnarray}
{\rm Re} & = & {{\rho_\infty U_\infty L} \over {\mu_\infty}},
    \quad \quad \phantom{xxx} \mbox{\rm Reynolds number}, \\
{\rm Pr} & = &  {{C_{p,\infty} \mu_\infty} \over {k_\infty}},
    \quad \quad \phantom{xxx} \mbox{\rm Prandtl number}, \\
{\rm Fr}_i & = & {{u_\infty^2} \over {g_i L}},
    \quad \quad \phantom{xxxxxxi} \mbox{\rm Froude number}, \quad g_i \ne 0, \\
{\rm Ma} & = & \sqrt{{u^2_\infty} \over {\gamma R T_\infty /W}},
    \quad \quad \mbox{\rm Mach number},
\end{eqnarray}
%
where $\gamma$ is the ratio of specific heats.

For small Mach numbers, ${\rm Ma} \ll 1$, the kinetic energy, viscous
work, and gravity work terms can be neglected in the energy equation
since those terms are scaled by the square of the Mach number.  The
inverse of Mach number squared remains in the momentum equations,
suggesting singular behavior. In order to explore the singularity, 
the pressure, velocity and temperature are expanded as asymptotic
series in terms of the parameter $\epsilon$:
%
\begin{eqnarray}
   \bar{P} & = & \bar{P}_0     + \bar{P}_1 \epsilon     + \bar{P}_2 \epsilon^2 \ldots \\
 \bar{u}_i & = & \bar{u}_{i,0} + \bar{u}_{i,1} \epsilon + \bar{u}_{i,2} \epsilon^2 \ldots \\
   \bar{T} & = & \bar{T}_0     + \bar{T}_1 \epsilon     + \bar{T}_2 \epsilon^2 \ldots
\end{eqnarray}
%
The zeroeth-order terms are collected together in each of the
equations.  The form of the continuity equation stays the same. 
The gradient of the pressure in the zeroeth-order
momentum equations can become singular since it is divided by the 
characteristic Mach number squared.  In order for the zeroeth-order
momentum equations to remain well-behaved, the spatial variation of 
the $\bar{P}_0$ term must be zero.  If the magnitude of the expansion 
parameter is selected to be proportional to the square of the characteristic 
Mach number, $\epsilon = \gamma {\rm Ma}^2$,
then the $\bar{P}_1$ term can be included in the zeroeth-order
momentum equation.
%
\begin{equation}
   {1 \over {\gamma {\rm Ma}^2}}
   {{\partial \bar{P}} \over {\partial x_i}}  =
   {{\partial} \over {\partial x_i}} \left( {1 \over {\gamma {\rm Ma}^2}} \bar{P}_0
       + {\epsilon \over {\gamma {\rm Ma}^2}} \bar{P}_1 + \ldots \right) =
   {{\partial} \over {\partial x_i}} \left( \bar{P}_1 + \epsilon \bar{P}_2 + \ldots 
   \phantom{1 \over {\gamma {\rm Ma}^2}} \right)
\end{equation}
%
The form of the energy equation remains the same, less the kinetic
energy, viscous work and gravity work terms.  The $P_0$
term remains in the energy equation as a time derivative.  
The low Mach number equations are the zeroeth-order equations
in the expansion including the $P_1$ term in the momentum
equations. The expansion results in two different types of pressure and
they are considered to be split into a thermodynamic component
and a dynamic component.  The thermodynamic pressure is constant in space,
but can change in time.  The thermodynamic pressure is used in the
equation of state.  The dynamic pressure only arises as a gradient
term in the momentum equation and acts to enforce continuity. 
The unsplit dimensional pressure is
%
\begin{equation}
  P = P_{th} + \gamma {\rm Ma}^2 P_1,
\end{equation}
%
where the dynamic pressure, $p=P-P_{th}$, is related to a pressure coefficient
%
\begin{equation}
  \bar{P}_1 = {{P - P_{th}} \over {\rho_\infty u^2_\infty}} P_{th}.
\end{equation}
%
The resulting unscaled low Mach number equations are:
%
\begin{eqnarray}
{{\partial \rho} \over {\partial t}} + {{\partial \rho u_j} 
                               \over {\partial x_j}} & =  & 0, \label{lmcon} \\
{{\partial \rho u_i} \over {\partial t}} + {{\partial \rho u_j u_i}
    \over {\partial x_j}} + {{\partial P} \over {\partial x_i}} & = &
    {{\partial \tau_{ij}} \over {\partial x_j}} 
     +  \left( \rho - \rho_{\circ} \right) g_i, \label{lmmom} \\
{{\partial \rho h} \over {\partial t}} + {{\partial \rho u_j h}
    \over {\partial x_j}}  & = &
    - {{\partial q_j} \over {\partial x_j}}
    + {{\partial P_{th}} \over {\partial t}}, \label{lmenrg}
\end{eqnarray}
%
where the ideal gas law becomes
%
\begin{equation}
P_{th}  =  \rho {R \over W} T.
\end{equation}
%
The hydrostatic pressure gradient has been subtracted
from the momentum equation, assuming an ambient density
of $\rho_{\circ}$.  The stress tensor and heat conduction 
remain the same as in the original equations.
